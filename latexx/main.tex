% Tệp: main.tex
%%%%%%%%%%%%%%%%%%%%%%%%%%%%%%%%%%%%%%%%%%%%%%%%%%%%%%%
% --- CẤU HÌNH CƠ BẢN ---
%%%%%%%%%%%%%%%%%%%%%%%%%%%%%%%%%%%%%%%%%%%%%%%%%%%%%%%
% Dùng cỡ chữ 13pt (extreport hỗ trợ tốt các cỡ chữ lạ)
\documentclass[13pt, a4paper, oneside]{extreport}

% Gói hỗ trợ tiếng Việt và Font Times New Roman
\usepackage[utf8]{inputenc}
\usepackage[T5]{fontenc}
\usepackage[vietnamese]{babel}
\usepackage{newtxtext, newtxmath} % Font Times

% Gói đồ họa và bảng biểu
\usepackage{graphicx}
\usepackage{tabularx}
\usepackage{booktabs}
\usepackage{indentfirst} % Thụt đầu dòng đoạn đầu tiên

\usepackage{enumitem} % Để chỉnh format danh sách
%%%%%%%%%%%%%%%%%%%%%%%%%%%%%%%%%%%%%%%%%%%%%%%%%%%%%%%
% --- ĐỊNH DẠNG TRANG (PAGE LAYOUT) ---
%%%%%%%%%%%%%%%%%%%%%%%%%%%%%%%%%%%%%%%%%%%%%%%%%%%%%%%
% Yêu cầu: Trái 3.5, Phải 2, Trên 3.5, Dưới 3
\usepackage[
    top=3.5cm,
    bottom=3cm,
    left=3.5cm,
    right=2cm
]{geometry}

% Yêu cầu: Dãn dòng 1.5 lines
\usepackage{setspace}
\onehalfspacing 

% Yêu cầu: Before 0pt, After 0pt (Không giãn đoạn)
\setlength{\parskip}{0pt}
\setlength{\parindent}{1.27cm} % Thụt đầu dòng chuẩn

%%%%%%%%%%%%%%%%%%%%%%%%%%%%%%%%%%%%%%%%%%%%%%%%%%%%%%%
% --- ĐỊNH DẠNG TIÊU ĐỀ (TITLES) ---
%%%%%%%%%%%%%%%%%%%%%%%%%%%%%%%%%%%%%%%%%%%%%%%%%%%%%%%
% Sử dụng gói titlesec để tùy biến sâu hơn
\usepackage{titlesec}

% 1. Tiêu đề Cấp 1 (CHƯƠNG): Font 14, In hoa, Đậm, Căn giữa
\titleformat{\chapter}[display]
  {\centering\normalfont\bfseries\fontsize{14pt}{16.8pt}\selectfont} % Format
  {\chaptertitlename\ \thechapter} % Label (CHƯƠNG 1)
  {0pt} % Khoảng cách giữa Label và Title
  {\MakeUppercase} % In hoa tên chương

\titlespacing*{\chapter}{0pt}{-10pt}{20pt} % Căn chỉnh khoảng cách trên dưới chương

% 2. Tiêu đề Cấp 2 (1.1): Font 13, Thường, Đậm, Căn trái
\titleformat{\section}
  {\normalfont\bfseries\fontsize{13pt}{15.6pt}\selectfont}
  {\thesection}{1em}{}
  
\titlespacing*{\section}{0pt}{6pt}{0pt} % Before/After nhỏ gọn

% 3. Tiêu đề Cấp 3 (1.1.1): Font 13, Thường, Nghiêng, Căn trái
% Lưu ý: Bạn yêu cầu "nghiêng" và "in thường" (không đậm)
\titleformat{\subsection}
  {\normalfont\itshape\fontsize{13pt}{15.6pt}\selectfont}
  {\thesubsection}{1em}{}

\titlespacing*{\subsection}{0pt}{6pt}{0pt}

% 4. Tiêu đề Cấp 4 (a. Tên tiêu đề): Font 13, Thường, Căn trái
% Trong LaTeX, cấp dưới subsection là subsubsection
\titleformat{\subsubsection}
  {\normalfont\fontsize{13pt}{15.6pt}\selectfont}
  {\thesubsubsection}{1em}{}

% Đổi đánh số 1.1.1.1 thành a. b. c.
\renewcommand{\thesubsubsection}{\alph{subsubsection}.}

%%%%%%%%%%%%%%%%%%%%%%%%%%%%%%%%%%%%%%%%%%%%%%%%%%%%%%%
% --- HÌNH ẢNH, BẢNG BIỂU & CAPTION ---
%%%%%%%%%%%%%%%%%%%%%%%%%%%%%%%%%%%%%%%%%%%%%%%%%%%%%%%
\usepackage{caption}
\usepackage{float} 

% Cấu hình chung cho Caption: Font 13
\DeclareCaptionFont{size13}{\fontsize{13pt}{15.6pt}\selectfont}

% Yêu cầu Bảng: Tiêu đề trên, Căn trái, Đậm, After 6pt
\captionsetup[table]{
    position=top,
    justification=raggedright, % Căn trái
    singlelinecheck=false,     % Giữ căn trái ngay cả khi caption ngắn
    labelfont={bf, size13},    % Chữ "Bảng 1.1" đậm, 13pt
    textfont={bf, size13},     % Nội dung đậm, 13pt
    skip=6pt                   % Layout After 6 (khoảng cách đến bảng)
}

% Yêu cầu Hình: Tiêu đề dưới, Căn giữa, Đậm, Before 6pt
\captionsetup[figure]{
    position=bottom,
    justification=centering,   % Căn giữa
    labelfont={bf, size13},
    textfont={bf, size13},
    skip=6pt                   % Layout Before 6 (khoảng cách từ hình đến caption)
}

% Định nghĩa lệnh \mysource để ghi nguồn: Căn phải, nghiêng, font 12
\newcommand{\mysource}[1]{%
    \vspace{0pt} % Khoảng cách nhỏ với đối tượng phía trên
    \begin{flushright}
        \fontsize{12pt}{14.4pt}\selectfont \itshape Nguồn: #1
    \end{flushright}
    \vspace{6pt} % Layout After 6
}

% Đánh số theo chương (Hình 2.1, Bảng 3.2)
\usepackage{chngcntr}
\counterwithin{figure}{chapter}
\counterwithin{table}{chapter}

% Đổi tên mặc định
\addto\captionsvietnamese{%
  \renewcommand{\figurename}{Hình}%
  \renewcommand{\tablename}{Bảng}%
  \renewcommand{\chaptername}{CHƯƠNG}
}

%%%%%%%%%%%%%%%%%%%%%%%%%%%%%%%%%%%%%%%%%%%%%%%%%%%%%%%
% --- CÁC GÓI KHÁC & HEADER/FOOTER ---
%%%%%%%%%%%%%%%%%%%%%%%%%%%%%%%%%%%%%%%%%%%%%%%%%%%%%%%
\usepackage{fancyhdr}
\pagestyle{fancy}
\fancyhf{}
\fancyfoot[C]{\thepage} % Số trang ở giữa dưới
\renewcommand{\headrulewidth}{0pt}

% Gói vẽ khung bìa
\usepackage{tikz}
\usetikzlibrary{calc}

% Tài liệu tham khảo
\usepackage[backend=biber, style=numeric]{biblatex}
\addbibresource{refs.bib}
\usepackage[colorlinks=true, linkcolor=blue, urlcolor=cyan]{hyperref}

%%%%%%%%%%%%%%%%%%%%%%%%%%%%%%%%%%%%%%%%%%%%%%%%%%%%%%%
% --- BẮT ĐẦU TÀI LIỆU ---
%%%%%%%%%%%%%%%%%%%%%%%%%%%%%%%%%%%%%%%%%%%%%%%%%%%%%%%
\begin{document}

\begin{titlepage}
    % Tệp: sections/00_trangbia.tex
\begin{titlepage}
    % --- TẠO KHUNG VIỀN ---
    % Khung viền được căn chỉnh dựa trên lề:
    % Trái: 3.5cm -> Viền cách mép 3.0cm
    % Phải: 2.0cm -> Viền cách mép 1.5cm
    % Trên: 3.5cm -> Viền cách mép 3.0cm
    % Dưới: 3.0cm -> Viền cách mép 2.5cm
    \begin{tikzpicture}[overlay,remember picture]
        \draw [line width=3pt]
            ($ (current page.north west) + (3.0cm,-3.0cm) $)
            rectangle
            ($ (current page.south east) + (-1.5cm,2.5cm) $);
        \draw [line width=0.5pt]
            ($ (current page.north west) + (3.1cm,-3.1cm) $)
            rectangle
            ($ (current page.south east) + (-1.6cm,2.6cm) $);
    \end{tikzpicture}
    
    \centering
    % --- PHẦN ĐẦU ---
    % Dùng vspace âm để đẩy nội dung lên sát lề trên (nếu cần)
    \vspace*{-0.5cm}
    
    {\large \bfseries TRƯỜNG ĐẠI HỌC GIAO THÔNG VẬN TẢI TP.HCM \par}
    {\large \bfseries VIỆN CÔNG NGHỆ THÔNG TIN VÀ ĐIỆN, ĐIỆN TỬ \par}
    
    \vspace{1cm}
    
    % --- LOGO ---
    % Đường dẫn ảnh đã sửa cho phù hợp cấu trúc thư mục
    \includegraphics[width=5cm]{img/logoUTH.jpg}
    
    \vspace{1.5cm}
    
    % --- TÊN BÁO CÁO ---
    {\fontsize{24pt}{30pt}\selectfont \bfseries BÁO CÁO ĐỒ ÁN \par}
    \vspace{1cm}
    
    {\large \bfseries ĐỀ TÀI: \par}
    % Tên đề tài in hoa, đậm, cỡ chữ lớn
    {\fontsize{20pt}{24pt}\selectfont \bfseries STUSHARE - ỨNG DỤNG CHIA SẺ TÀI LIỆU \par}
    
    \vfill 
    
    % --- THÔNG TIN SINH VIÊN ---
    % Sử dụng bảng để căn chỉnh thẳng hàng
    \begin{table}[h]
        \centering
        \large
        \begin{tabular}{l l}
            \textbf{GVHD:} & \textbf{[Học hàm/Học vị] [Tên GVHD]} \\[0.3cm]
        \textbf{SVTH:} & \textbf{Đào Vũ Dũng} \\
                       & MSSV: 0672 0500 1514 \\[0.2cm]
                       & \textbf{Phạm Công Hoàng} \\
                       & MSSV: 0642 0500 4128 \\[0.2cm]
                       & \textbf{Nguyễn Quốc Thiện} \\
                       & MSSV: [Mã số SV 3] \\
        \end{tabular}
    \end{table}
    
    \vfill
    
    % --- NGÀY THÁNG ---
    {\large \bfseries TP. HỒ CHÍ MINH, NĂM 2025}
    \vspace{1cm}

\end{titlepage}
\end{titlepage}

\pagenumbering{roman} % Đánh số La Mã cho phần đầu

% Tệp: sections/02_bangphanchiacongviec.tex

\chapter*{BẢNG PHÂN CÔNG CÔNG VIỆC CỦA CÁC THÀNH VIÊN}
\addcontentsline{toc}{chapter}{Bảng phân công công việc}

\begin{center}
    \fontsize{13pt}{15.6pt}\selectfont
    
    \begin{tabularx}{\textwidth}{|m{4cm}|>{\raggedright\arraybackslash}X|}
        \hline
        \textbf{MSSV + Tên} & \multicolumn{1}{c|}{\textbf{NHIỆM VỤ}} \\
        \hline
        
        % --- THÀNH VIÊN 1 ---
        \centering 0672 0500 1514 \\ Đào Vũ Dũng     &
        \vspace{-0.5em}
        \begin{itemize}[leftmargin=*, nosep, label=-]
            \item Design Figma, Soạn thảo LaTeX.
            \item Chức năng: Upload tài liệu, Thông báo.
            \item Chức năng Cá nhân:
            \begin{itemize}[label=$\bullet$, leftmargin=1em]
                \item Bảng xếp hạng.
                \item Quản lý bài: Đã đăng, Đã lưu, Đã tải xuống.
            \end{itemize}
            \item Chức năng Setting:
            \begin{itemize}[label=$\bullet$, leftmargin=1em]
                \item Tài khoản \& Bảo mật.
                \item Bật thông báo.
                \item Giao diện \& Ngôn ngữ.
                \item Khác: Về ứng dụng, Liên hệ, Báo cáo, Chuyển TK, Đăng xuất.
            \end{itemize}
        \end{itemize}
        \vspace{0.5em}
        \\
        \hline
        
        % --- THÀNH VIÊN 2 ---
        \centering 0642 0500 4128 \\ Phạm Công Hoàng &
        \vspace{-0.5em}
        \begin{itemize}[leftmargin=*, nosep, label=-]
            \item Chức năng Intro (logo, màn hình giới thiệu).
            \item Đăng nhập, Đăng ký (Mail, SĐT, Google).
            \item Quên mật khẩu (qua email).
            \item Design Figma.
            \item Soạn PPT + báo cáo LaTeX.
            \item Thuyết trình.
        \end{itemize}
        \vspace{0.5em}
        \\
        \hline
        
        % --- THÀNH VIÊN 3 ---
        \centering 0642 0500 4659 \\ Nguyễn Quốc Thiện &
        \vspace{-0.5em}
        \begin{itemize}[leftmargin=*, nosep, label=-]
            \item Design Figma.
            \item Chức năng của Màn hình chính: Danh sách tài liệu, tài liệu mới được tải lên hoặc theo dạng.
            \item Chức năng Tìm kiếm tài liệu: Kết quả tiềm kiếm(khi có hoặc không có kết quả)
            \item Chức năng Xem tài liệu: Xem, tải về, lưu.
            \item Chức năng Admin: Ban người dùng, thông báo report, thông báo toàn sever.
            \item Chức năng của cộng đồng hỏi đáp tài liệu.
        \end{itemize}
        \vspace{0.5em}
        \\
        \hline
    \end{tabularx}
\end{center}

\titleformat{\section}
  {\fontsize{14pt}{18pt}\bfseries}
  {}{0em}{}
  [\titlerule]


\section*{DỰ ÁN: STUSHARE - ỨNG DỤNG CHIA SẺ TÀI LIỆU}
\vspace{0.4cm}

\subsection*{Mô Tả Tổng Quan}
StuShare là ứng dụng di động hỗ trợ sinh viên tìm kiếm, chia sẻ và tải xuống tài liệu học tập, đề thi và bài giảng.  
Ứng dụng tạo ra một môi trường kết nối, giúp sinh viên hỗ trợ nhau thông qua việc chia sẻ tài liệu, yêu cầu tài liệu và hệ thống bảng xếp hạng đóng góp.

\begin{center}
    \includegraphics[width=0.7\linewidth]{img/mau.png}
\end{center}

\subsection*{Tính Năng Chính}
\begin{itemize}[leftmargin=1.2cm]
    \item {Xác thực người dùng:}  
    Đăng ký, đăng nhập (Email/Mật khẩu, Số điện thoại).  
    Quên mật khẩu, xác thực OTP.

    \item {Trang chủ \& Hiển thị tài liệu:}  
    Hiển thị danh sách tài liệu mới, tài liệu đề thi.  
    Xem chi tiết tài liệu.

    \item {Tương tác \& Tiện ích:}  
    Tìm kiếm theo từ khóa hoặc tag.  
    Tải xuống tài liệu qua \textit{DownloadWorker}.  
    Upload tài liệu lên hệ thống.

    \item {Cộng đồng:}  
    Tạo và xem dánh sách yêu cầu tìm tài liệu từ người dùng khác.  
    Bảng xếp hạng xếp hạng thành viên theo mức độ đóng góp.

    \item {Cá nhân \& Cài đặt:}  
    Quản lý hồ sơ, đổi mật khẩu, bảo mật tài khoản.  
    Cài đặt giao diện, thông báo.  
    Nhận thông báo cập nhật và tương tác.

    \item {Quản trị hệ thống:}  
    Dashboard theo dõi thống kê tổng quan của ứng dụng. Tiếp nhận và xử lý các báo cáo vi phạm nội dung. Quản lý danh sách người dùng và gửi thông báo hệ thống.
\end{itemize}

\subsection*{Công Nghệ Sử Dụng}
\begin{itemize}[leftmargin=1.2cm]
    \item Ngôn ngữ lập trình: Kotlin 
   \item Giao diện: Jetpack Compose 
   \item {Kiến trúc:} Model-View-ViewModel + Clean Architecture 
    \item {Dependency Injection:} Hilt 
   \item {Xử lý bất đồng bộ:} Coroutines \& Flow 
   \item {Lưu trữ cục bộ:} Room Database 
   \item {Kết nối mạng:} Retrofit 2 
   \item {Tác vụ nền:} WorkManager
   \item {Backend/Cloud:} Firebase Authentication, Storage, Analytics
\end{itemize}

\subsection*{Cách Cài Đặt}
\begin{itemize}[leftmargin=1.2cm]
    \item Clone mã nguồn: Tải source code từ GitHub về máy.
    \item Cấu hình Firebase: Tạo dự án trên Firebase Console. Kích hoạt các dịch vụ: Authentication, Firestore Database, Storage. Tải file google-services.json từ Firebase Console và đặt vào thư mục app/ của dự án.
    \item Đồng bộ hóa (Sync Gradle): Mở Android Studio, chọn File > Sync Project with Gradle Files để tải các thư viện (Hilt, Room, Coil, v.v.).
    \item Chạy ứng dụng: Kết nối thiết bị thật hoặc mở Emulator, nhấn nút Run.
\end{itemize}

\subsection*{Định Hướng Phát Triển}
\begin{itemize}[leftmargin=1.2cm]
    \item Tích hợp AI kiểm duyệt: Nâng cấp module Admin hiện tại bằng cách tích hợp AI để tự động quét và cảnh báo các tài liệu chứa nội dung không phù hợp hoặc vi phạm bản quyền trước khi cần Admin duyệt thủ công.
    
    \item {Tối ưu giao diện:} Nâng cấp UI/UX cho trải nghiệm hiện đại hơn.
    \item {Cải thiện tìm kiếm:} Tìm kiếm nâng cao theo nhiều tiêu chí, gợi ý thông minh.
    \item Bình luận, Đánh giá chi tiết: Bổ sung tính năng cho phép người dùng bình luận dưới mỗi tài liệu, thay vì chỉ xếp hạng sao như hiện tại.
\end{itemize}

\vspace{0.5cm}
\subsection*{Link Git Dự Án}
\noindent\url{https://github.com/Dungdao2309/MOBILE}


% Tệp: sections/02_camdoan.tex
\chapter*{LỜI CAM ĐOAN}
% Thêm "Lời cam đoan" vào Mục lục
\addcontentsline{toc}{chapter}{Lời cam đoan}

Tôi xin cam đoan đây là công trình nghiên cứu của riêng tôi. Các số liệu, kết quả nêu trong Báo cáo thực tập tốt nghiệp là trung thực, xuất phát từ tình hình thực tế của đơn vị thực tập. Các tài liệu trích dẫn đã được chỉ rõ nguồn.

\vspace{5cm}

\begin{flushright}
    \textit{[Thành phố], ngày... tháng... năm...} \\
    \textbf{Sinh viên thực hiện} \\
    \vspace{2cm} \\
    \textbf{[Họ và tên của bạn]}
\end{flushright}
\tableofcontents
% Tệp: sections/03_vietat.tex
\chapter*{DANH MỤC CÁC TỪ VIẾT TẮT}
\addcontentsline{toc}{chapter}{Danh mục các từ viết tắt}

% Sử dụng bảng để căn chỉnh
% Yêu cầu: Sắp xếp theo thứ tự ABC
\begin{tabular}{p{3cm} p{11cm}}
    \toprule
    \textbf{Viết tắt} & \textbf{Diễn giải} \\
    \midrule
    ABC & Một cụm từ ví dụ \\
    CNTT & Công nghệ Thông tin \\
    TTTN & Thực tập Tốt nghiệp \\
    XYZ & Một cụm từ khác \\
    \bottomrule
\end{tabular}

\pagenumbering{arabic} % Đánh số 1, 2, 3 cho nội dung chính

% Tệp: sections/04_mordau.tex
\chapter*{LỜI MỞ ĐẦU}
\addcontentsline{toc}{chapter}{Lời mở đầu}

% Yêu cầu: Sử dụng danh sách có thứ tự (1, 2, 3...)
\begin{enumerate}
    \item \textbf{Tính cấp thiết của đề tài}
    
    (Đặt vấn đề, tầm quan trọng, ý nghĩa của đề tài, lý do chọn đề tài...)
    \lipsum[1] % Xóa \lipsum[1] và viết nội dung của bạn vào đây

    \item \textbf{Tình hình nghiên cứu}
    
    (Tóm tắt về những đề tài, công trình nghiên cứu đã công bố có liên quan...)
    \lipsum[2]

    \item \textbf{Mục đích nghiên cứu}
    
    (Đề tài nhằm giải quyết vấn đề gì, hướng tới kết quả gì...)
    \lipsum[3]

    \item \textbf{Nhiệm vụ nghiên cứu}
    
    (Đề tài sẽ thực hiện những nghiên cứu, những nhiệm vụ cụ thể gì...)

    \item \textbf{Phương pháp nghiên cứu}
    
    (Sử dụng phương pháp thu thập thông tin nào, phương pháp nào để nghiên cứu...)

    \item \textbf{Các kết quả đạt được của đề tài}
    
    (Tóm tắt kết quả)

    \item \textbf{Kết cấu của TTTN}
    
    (Báo cáo TTTN gồm có 3 chương: \par
    Chương 1: [Tên chương 1]. \par
    Chương 2: [Tên chương 2]. \par
    Chương 3: [Tên chương 3].)
\end{enumerate}
% Tệp: sections/05_chuong1.tex
\chapter{GIỚI THIỆU TỔNG QUAN VỀ [ĐƠN VỊ THỰC TẬP]}

\section{Mục 1.1 (Thường, Đậm)}
Nội dung của mục 1.1. Văn bản tham khảo \cite{nguyenvanA2001}.
Như Boulding \cite[tr. 50]{boulding1955} đã chỉ ra...

    \subsection{Nhóm tiểu mục 1.1.1 (Nghiêng, Đậm)}
    Nội dung của nhóm tiểu mục 1.1.1.
    
        \subsubsection{Tiểu mục 1.1.1.1 (Nghiêng)}
        Nội dung của tiểu mục 1.1.1.1.
        
        \subsubsection{Tiểu mục 1.1.1.2 (Nghiêng)}
        Nội dung của tiểu mục 1.1.1.2.

% --- VÍ DỤ VỀ BẢNG ---
\begin{table}[h!]
    % Caption (tiêu đề) phải ở trên
    \caption{Đây là tiêu đề của Bảng 1.1}
    \label{tab:bang_vi_du}
    \centering
    \begin{tabular}{l l r}
        \toprule
        \textbf{Cột 1} & \textbf{Cột 2} & \textbf{Cột 3 (Số)} \\
        \midrule
        Dữ liệu A & Dữ liệu B & 100 \\
        Dữ liệu C & Dữ liệu D & 200 \\
        \bottomrule
    \end{tabular}
    \par % Thêm một dòng mới
    \textit{Nguồn: Tổng cục thống kê (2010)}
\end{table}

\section{Mục 1.2 (Thường, Đậm)}
Kết quả được thể hiện trong Bảng \ref{tab:bang_vi_du} và Hình \ref{fig:hinh_vi_du}.

% --- VÍ DỤ VỀ HÌNH ẢNH ---
\begin{figure}[h!]
    \centering
    % \includegraphics[width=0.8\textwidth]{ten_file_hinh_anh.jpg}
    % (Nếu không có hình, dùng placeholder)
    \fbox{\rule{0pt}{5cm} \rule{0.8\textwidth}{0pt}}
    
    % Caption (tiêu đề) phải ở dưới
    \caption{Đây là tiêu đề của Hình 1.1}
    \label{fig:hinh_vi_du}
\end{figure}
% Tệp: sections/07_chuong2.tex

\chapter{PHÂN TÍCH VÀ THIẾT KẾ HỆ THỐNG}
\label{ch:phantichthietke}

Chương này trình bày chi tiết về các yêu cầu chức năng và phi chức năng của hệ thống, đồng thời mô tả các bản thiết kế về Use Case, luồng hoạt động, tương tác và cơ sở dữ liệu của ứng dụng StuShare.

\section{Phân tích yêu cầu hệ thống}

\subsection{Yêu cầu chức năng} 
Dựa trên khảo sát nhu cầu thực tế của sinh viên, hệ thống StuShare cần đáp ứng các nhóm chức năng chính sau:

\begin{itemize}
    \item \textbf{Nhóm chức năng Xác thực \& Tài khoản:}
    \begin{itemize}
        \item Đăng ký tài khoản mới (Email, Số điện thoại).
        \item Đăng nhập hệ thống (Email/Pass, Google Sign-In).
        \item Quên mật khẩu và khôi phục qua Email.
        \item Cập nhật thông tin cá nhân (Avatar, Tên, Chuyên ngành).
        \item Đổi mật khẩu và bảo mật tài khoản.
    \end{itemize}

    \item \textbf{Nhóm chức năng Quản lý Tài liệu:}
    \begin{itemize}
        \item Xem danh sách tài liệu mới nhất, tài liệu đề thi.
        \item Xem chi tiết thông tin tài liệu (Tên, mô tả, tác giả, lượt tải).
        \item Tải lên tài liệu mới (Upload file PDF, Word) kèm tiêu đề và thẻ (Tag).
        \item Tải xuống tài liệu về thiết bị.
    \end{itemize}

    \item \textbf{Nhóm chức năng Tương tác \& Tìm kiếm:}
    \begin{itemize}
        \item Tìm kiếm tài liệu theo từ khóa.
        \item Lọc tài liệu theo thẻ (Tag).
        \item Xem bảng xếp hạng thành viên tích cực.
        \item Gửi yêu cầu tìm tài liệu (Request Document).
    \end{itemize}
\end{itemize}

\subsection{Yêu cầu phi chức năng}
\begin{itemize}
    \item \textbf{Hiệu năng:} Ứng dụng phản hồi nhanh, thời gian tải danh sách tài liệu dưới 2 giây trong điều kiện mạng ổn định.
    \item \textbf{Bảo mật:} Mật khẩu người dùng được mã hóa. Sử dụng Token để xác thực phiên làm việc. Dữ liệu cá nhân được bảo vệ theo quy định.
    \item \textbf{Giao diện (UI/UX):} Thiết kế hiện đại, thân thiện, dễ sử dụng, hỗ trợ giao diện Sáng/Tối (Light/Dark Mode).
    \item \textbf{Độ tin cậy:} Hệ thống hoạt động ổn định 24/7, có cơ chế xử lý lỗi khi mất kết nối mạng (Offline Mode cơ bản).
\end{itemize}

\section{Thiết kế Use Case}

\subsection{Sơ đồ Use Case tổng quát}
Sơ đồ dưới đây mô tả tổng quan các tác nhân và chức năng của hệ thống StuShare.

\begin{figure}[H]
    \centering
    % height=0.4\textheight: Giới hạn chiều cao để 1 trang chứa được 2 hình
    \includegraphics[width=1.1\textwidth, height=0.6\textheight, keepaspectratio]{img/usecase/usecase2.png} 
    \caption{Sơ đồ Use Case tổng quát hệ thống StuShare}
    \label{fig:usecase_tongquat}
\end{figure}

\subsection{Đặc tả các Use Case chính}

\begin{table}[H]
    \centering
    \caption{Đặc tả Use Case "Đăng nhập"}
    \begin{tabular}{|p{4cm}|p{10cm}|}
        \hline
        \textbf{Tên Use Case} & \textbf{Đăng nhập hệ thống} \\
        \hline
        \textbf{Tác nhân} & Sinh viên (User) \\
        \hline
        \textbf{Mục đích} & Cho phép người dùng truy cập vào các chức năng yêu cầu xác thực. \\
        \hline
        \textbf{Luồng sự kiện chính} & 
        1. User mở ứng dụng, chọn chức năng Đăng nhập. \newline
        2. Hệ thống hiển thị form nhập Email/Mật khẩu. \newline
        3. User nhập thông tin và nhấn "Đăng nhập". \newline
        4. Hệ thống kiểm tra thông tin qua Firebase Auth. \newline
        5. Nếu đúng, chuyển hướng vào Trang chủ. \\
        \hline
        \textbf{Ngoại lệ} & Nhập sai mật khẩu quá 5 lần -> Khóa tạm thời. \\
        \hline
    \end{tabular}
\end{table}

\begin{table}[H]
    \centering
    \caption{Đặc tả Use Case "Tải lên tài liệu"}
    \begin{tabular}{|p{4cm}|p{10cm}|}
        \hline
        \textbf{Tên Use Case} & \textbf{Tải lên tài liệu} \\
        \hline
        \textbf{Tác nhân} & Sinh viên (User) \\
        \hline
        \textbf{Điều kiện tiên quyết} & User đã đăng nhập thành công. \\
        \hline
        \textbf{Luồng sự kiện chính} & 
        1. User chọn nút "Upload" từ màn hình chính. \newline
        2. User chọn file từ thiết bị và điền thông tin (Tiêu đề, Mô tả). \newline
        3. User nhấn "Đăng tải". \newline
        4. Hệ thống upload file lên Storage và lưu metadata vào Database. \newline
        5. Thông báo thành công và cộng điểm cho User. \\
        \hline
    \end{tabular}
\end{table}

\section{Thiết kế luồng hoạt động (Activity Diagram)}
Dưới đây là các sơ đồ hoạt động mô tả quy trình nghiệp vụ của 9 chức năng chính trong hệ thống.

% 2.3.1
\subsection{Sơ đồ hoạt động chức năng Đăng ký}
\begin{figure}[H] 
    \centering 
    \includegraphics[width=0.85\textwidth, height=0.4\textheight, keepaspectratio]{img/activity/dangki.png} 
    \caption{Activity Diagram - Đăng ký tài khoản} 
\end{figure}

% 2.3.2
\subsection{Sơ đồ hoạt động chức năng Đăng nhập}
\begin{figure}[H] 
    \centering 
    \includegraphics[width=0.85\textwidth, height=0.4\textheight, keepaspectratio]{img/activity/dangnhap.png} 
    \caption{Activity Diagram - Đăng nhập} 
\end{figure}

% 2.3.3
\subsection{Sơ đồ hoạt động chức năng Tìm kiếm tài liệu}
\begin{figure}[H] 
    \centering 
    \includegraphics[width=0.85\textwidth, height=0.4\textheight, keepaspectratio]{img/activity/timkiemtailieu.png} 
    \caption{Activity Diagram - Tìm kiếm tài liệu} 
\end{figure}

% 2.3.4
\subsection{Sơ đồ hoạt động chức năng Báo cáo vi phạm}
\begin{figure}[H] 
    \centering 
    \includegraphics[width=0.85\textwidth, height=0.4\textheight, keepaspectratio]{img/activity/baocaovipham.png} 
    \caption{Activity Diagram - Báo cáo vi phạm} 
\end{figure}

% 2.3.5
\subsection{Sơ đồ hoạt động chức năng Tải xuống tài liệu}
\begin{figure}[H] 
    \centering 
    \includegraphics[width=0.85\textwidth, height=0.4\textheight, keepaspectratio]{img/activity/taivetailieu.png} 
    \caption{Activity Diagram - Tải xuống tài liệu} 
\end{figure}

% 2.3.6
\subsection{Sơ đồ hoạt động chức năng Tải lên tài liệu}
\begin{figure}[H] 
    \centering 
    \includegraphics[width=0.85\textwidth, height=0.4\textheight, keepaspectratio]{img/activity/tailentailieu.png} 
    \caption{Activity Diagram - Tải lên tài liệu} 
\end{figure}

% 2.3.7
\subsection{Sơ đồ hoạt động chức năng Cập nhật bảng xếp hạng}
\begin{figure}[H] 
    \centering 
    \includegraphics[width=0.85\textwidth, height=0.4\textheight, keepaspectratio]{img/activity/capnhatbangxemhang.png} 
    \caption{Activity Diagram - Cập nhật bảng xếp hạng} 
\end{figure}

% 2.3.8
\subsection{Sơ đồ hoạt động chức năng Tạo yêu cầu tài liệu}
\begin{figure}[H] 
    \centering 
    \includegraphics[width=0.85\textwidth, height=0.4\textheight, keepaspectratio]{img/activity/yeucautailieu.png} 
    \caption{Activity Diagram - Tạo yêu cầu tài liệu} 
\end{figure}

% 2.3.9
\subsection{Sơ đồ hoạt động chức năng Quên mật khẩu}
\begin{figure}[H] 
    \centering 
    \includegraphics[width=0.85\textwidth, height=0.4\textheight, keepaspectratio]{img/activity/quenmatkhau.png} 
    \caption{Activity Diagram - Quên mật khẩu} 
\end{figure}


\section{Thiết kế tương tác (Sequence Diagram)}
Mục này trình bày chi tiết sự tương tác giữa các đối tượng trong hệ thống thông qua các sơ đồ tuần tự.

% 1. Báo cáo vi phạm
\subsection{Sơ đồ tuần tự Báo cáo vi phạm}
\begin{figure}[H] 
    \centering 
    \includegraphics[width=0.85\textwidth, height=0.4\textheight, keepaspectratio]{img/Sequence/baocaovipham.png} 
    \caption{Sequence Diagram - Báo cáo vi phạm} 
\end{figure}

% 2. Bỏ lưu tài liệu
\subsection{Sơ đồ tuần tự Bỏ lưu tài liệu}
\begin{figure}[H] 
    \centering 
    \includegraphics[width=0.85\textwidth, height=0.4\textheight, keepaspectratio]{img/Sequence/boluutailieu.png} 
    \caption{Sequence Diagram - Bỏ lưu tài liệu} 
\end{figure}

% 3. Chỉnh sửa thông tin
\subsection{Sơ đồ tuần tự Chỉnh sửa thông tin}
\begin{figure}[H] 
    \centering 
    \includegraphics[width=0.85\textwidth, height=0.4\textheight, keepaspectratio]{img/Sequence/chinhsuathongtin.png} 
    \caption{Sequence Diagram - Chỉnh sửa thông tin} 
\end{figure}

% 4. Đăng ký thành công
\subsection{Sơ đồ tuần tự Đăng ký thành công}
\begin{figure}[H] 
    \centering 
    \includegraphics[width=0.85\textwidth, height=0.4\textheight, keepaspectratio]{img/Sequence/dangkithanhcong.png} 
    \caption{Sequence Diagram - Đăng ký thành công} 
\end{figure}

% 5. Đăng nhập thành công
\subsection{Sơ đồ tuần tự Đăng nhập thành công}
\begin{figure}[H] 
    \centering 
    \includegraphics[width=0.85\textwidth, height=0.4\textheight, keepaspectratio]{img/Sequence/dangnhapthanhcong.png} 
    \caption{Sequence Diagram - Đăng nhập thành công} 
\end{figure}

% 6. Đăng nhập thất bại
\subsection{Sơ đồ tuần tự Đăng nhập thất bại}
\begin{figure}[H] 
    \centering 
    \includegraphics[width=0.85\textwidth, height=0.4\textheight, keepaspectratio]{img/Sequence/dangnhapthatbai.png} 
    \caption{Sequence Diagram - Đăng nhập thất bại} 
\end{figure}

% 7. Đăng xuất
\subsection{Sơ đồ tuần tự Đăng xuất}
\begin{figure}[H] 
    \centering 
    \includegraphics[width=0.85\textwidth, height=0.4\textheight, keepaspectratio]{img/Sequence/dangxuat.png} 
    \caption{Sequence Diagram - Đăng xuất} 
\end{figure}

% 8. Đổi mật khẩu
\subsection{Sơ đồ tuần tự Đổi mật khẩu}
\begin{figure}[H] 
    \centering 
    \includegraphics[width=0.85\textwidth, height=0.4\textheight, keepaspectratio]{img/Sequence/doimk.png} 
    \caption{Sequence Diagram - Đổi mật khẩu} 
\end{figure}

% 9. Lưu tài liệu
\subsection{Sơ đồ tuần tự Lưu tài liệu}
\begin{figure}[H] 
    \centering 
    \includegraphics[width=0.85\textwidth, height=0.4\textheight, keepaspectratio]{img/Sequence/luutailieu.png} 
    \caption{Sequence Diagram - Lưu tài liệu} 
\end{figure}

% 10. Quên mật khẩu
\subsection{Sơ đồ tuần tự Quên mật khẩu}
\begin{figure}[H] 
    \centering 
    \includegraphics[width=0.85\textwidth, height=0.4\textheight, keepaspectratio]{img/Sequence/quyenmk.png} 
    \caption{Sequence Diagram - Quên mật khẩu} 
\end{figure}

% 11. Tải tài liệu thành công
\subsection{Sơ đồ tuần tự Tải tài liệu thành công}
\begin{figure}[H] 
    \centering 
    \includegraphics[width=0.85\textwidth, height=0.4\textheight, keepaspectratio]{img/Sequence/tailenthanhcong.png} 
    \caption{Sequence Diagram - Tải tài liệu thành công} 
\end{figure}

% 12. Tải tài liệu không thành công
\subsection{Sơ đồ tuần tự Tải tài liệu không thành công}
\begin{figure}[H] 
    \centering 
    \includegraphics[width=0.85\textwidth, height=0.4\textheight, keepaspectratio]{img/Sequence/tailen0thanhcong.png} 
    \caption{Sequence Diagram - Tải tài liệu không thành công} 
\end{figure}

% 13. Tải về tài liệu
\subsection{Sơ đồ tuần tự Tải về tài liệu}
\begin{figure}[H] 
    \centering 
    \includegraphics[width=0.85\textwidth, height=0.4\textheight, keepaspectratio]{img/Sequence/taivetailieu.png} 
    \caption{Sequence Diagram - Tải về tài liệu} 
\end{figure}

% 14. Tìm kiếm - có kết quả
\subsection{Sơ đồ tuần tự Tìm kiếm - có kết quả}
\begin{figure}[H] 
    \centering 
    \includegraphics[width=0.85\textwidth, height=0.4\textheight, keepaspectratio]{img/Sequence/timkiemcokq.png} 
    \caption{Sequence Diagram - Tìm kiếm có kết quả} 
\end{figure}

% 15. Yêu cầu tài liệu
\subsection{Sơ đồ tuần tự Yêu cầu tài liệu}
\begin{figure}[H] 
    \centering 
    \includegraphics[width=0.85\textwidth, height=0.4\textheight, keepaspectratio]{img/Sequence/yeucautailieu.png} 
    \caption{Sequence Diagram - Yêu cầu tài liệu} 
\end{figure}


\section{Thiết kế Cơ sở dữ liệu}

\subsection{Sơ đồ quan hệ thực thể (ERD)}
Mô hình dữ liệu của hệ thống bao gồm các thực thể chính: Người dùng, Tài liệu, Thông báo và Yêu cầu tài liệu.
\begin{figure}[H]
    \centering
    \includegraphics[width=0.9\textwidth, height=0.4\textheight, keepaspectratio]{img/db1.jpg} 
    \caption{Sơ đồ quan hệ thực thể (ERD)}
\end{figure}

\subsection{Mô tả chi tiết các bảng dữ liệu}
Dựa trên kiến trúc Room Database và Firebase Firestore đã cài đặt, dưới đây là chi tiết các bảng:

\subsubsection{Bảng Người dùng (UserEntity / users collection)}
Lưu trữ thông tin cá nhân và vai trò của người dùng trong hệ thống.
\begin{table}[H]
    \centering
    \begin{tabular}{|l|l|l|p{5cm}|}
        \hline
        \textbf{Tên trường} & \textbf{Kiểu dữ liệu} & \textbf{Ràng buộc} & \textbf{Mô tả} \\
        \hline
        id & String & PK & Mã định danh duy nhất (UID từ Firebase) \\
        \hline
        email & String & Unique, Not Null & Địa chỉ email đăng nhập \\
        \hline
        fullName & String & Not Null & Họ và tên hiển thị \\
        \hline
        studentId & String & Nullable & Mã số sinh viên \\
        \hline
        avatarUrl & String & Nullable & Đường dẫn ảnh đại diện \\
        \hline
        points & Int & Default 0 & Điểm thưởng đóng góp \\
        \hline
    \end{tabular}
    \mysource{Mã nguồn lớp UserEntity.kt}
\end{table}

\subsubsection{Bảng Tài liệu (Document / documents collection)}
Lưu trữ metadata của các tài liệu được chia sẻ.
\begin{table}[H]
    \centering
    \begin{tabular}{|l|l|l|p{5cm}|}
        \hline
        \textbf{Tên trường} & \textbf{Kiểu dữ liệu} & \textbf{Ràng buộc} & \textbf{Mô tả} \\
        \hline
        id & String & PK & Mã tài liệu \\
        \hline
        title & String & Not Null & Tiêu đề tài liệu \\
        \hline
        description & String & Nullable & Mô tả ngắn gọn \\
        \hline
        fileUrl & String & Not Null & Link tải file (Firebase Storage) \\
        \hline
        fileType & String & Not Null & Định dạng (PDF, DOCX...) \\
        \hline
        authorId & String & FK (User) & Người đăng tải \\
        \hline
        downloadCount & Int & Default 0 & Số lượt tải xuống \\
        \hline
    \end{tabular}
    \mysource{Mã nguồn lớp Document.kt}
\end{table}

\subsubsection{Bảng Thông báo (NotificationEntity)}
Lưu trữ các thông báo gửi đến người dùng.
\begin{table}[H]
    \centering
    \begin{tabular}{|l|l|l|p{5cm}|}
        \hline
        \textbf{Tên trường} & \textbf{Kiểu dữ liệu} & \textbf{Ràng buộc} & \textbf{Mô tả} \\
        \hline
        id & Int & PK, Auto Inc & Mã thông báo (Local DB) \\
        \hline
        title & String & Not Null & Tiêu đề thông báo \\
        \hline
        message & String & Not Null & Nội dung thông báo \\
        \hline
        isRead & Boolean & Default False & Trạng thái đã xem \\
        \hline
        timestamp & Long & Not Null & Thời gian nhận \\
        \hline
    \end{tabular}
    \mysource{Mã nguồn lớp NotificationEntity.kt}
\end{table}

\section{Thiết kế Giao diện}
Hệ thống sử dụng công nghệ \textbf{Jetpack Compose} để xây dựng giao diện người dùng (UI), đảm bảo tính hiện đại và mượt mà.
Các màn hình chính bao gồm:
\begin{itemize}
    \item \textbf{Màn hình Auth:} Login, Register với các trường nhập liệu và validate lỗi.
    \item \textbf{Màn hình Home:} Sử dụng LazyColumn để hiển thị danh sách tài liệu cuộn vô tận.
    \item \textbf{Màn hình Detail:} Hiển thị thông tin chi tiết và nút Download Floating Action Button.
\end{itemize}

\begin{figure}[H]
    \centering
    \includegraphics[width=0.9\textwidth, height=0.4\textheight, keepaspectratio]{img/mau.png} 
    \caption{Sơ đồ luồng màn hình (Screen Flow)}
\end{figure}
% Tệp: sections/08_chuong3.tex

\chapter{XÂY DỰNG VÀ KẾT QUẢ THỰC NGHIỆM}
\label{ch:thucnghiem}

Chương này trình bày chi tiết về môi trường phát triển, cách thức tổ chức mã nguồn dự án theo kiến trúc Clean Architecture và kết quả thực nghiệm thông qua các kịch bản kiểm thử trên ứng dụng StuShare.

\section{Môi trường cài đặt và Triển khai}

Để phát triển và vận hành ứng dụng StuShare, nhóm thực hiện đã sử dụng các công cụ phần cứng và phần mềm với cấu hình cụ thể như sau:

\subsection{Cấu hình phần cứng}
\begin{itemize}
    \item \textbf{Máy tính phát triển:} 
    \begin{itemize}
        \item CPU: Intel Core i5/i7 hoặc Apple M1/M2 trở lên.
        \item RAM: Tối thiểu 16GB (Khuyến nghị để chạy Android Studio và Emulator mượt mà).
        \item Ổ cứng: SSD 256GB trở lên.
    \end{itemize}
    \item \textbf{Thiết bị kiểm thử (Mobile):}
    \begin{itemize}
        \item Hệ điều hành: Android 7.0 (Nougat) trở lên (Tương ứng với API Level 24).
        \item Kết nối mạng: Wifi/4G ổn định để gọi API và Firebase.
    \end{itemize}
\end{itemize}

\subsection{Cấu hình phần mềm và Công cụ}
\begin{table}[h!]
    \centering
    \caption{Danh sách công cụ và phiên bản sử dụng}
    \begin{tabular}{|l|l|p{6cm}|}
        \hline
        \textbf{Công cụ/Thư viện} & \textbf{Phiên bản} & \textbf{Mục đích sử dụng} \\
        \hline
        Android Studio & Koala / Ladybug & Môi trường phát triển tích hợp (IDE). \\
        \hline
        Kotlin & 2.0.0 & Ngôn ngữ lập trình chính. \\
        \hline
        JDK & 17 & Java Development Kit yêu cầu bởi Gradle. \\
        \hline
        Gradle & 8.x & Công cụ quản lý dự án và build tự động. \\
        \hline
        Hilt & 2.51.1 & Dependency Injection (Tiêm phụ thuộc). \\
        \hline
        Firebase & BOM 33.1.0 & Authentication (Xác thực) và Backend cơ bản. \\
        \hline
    \end{tabular}
\end{table}

\section{Tổ chức mã nguồn}

Dự án được tổ chức theo kiến trúc \textbf{Clean Architecture} kết hợp với việc chia module theo tính năng (Package by Feature). Điều này giúp mã nguồn dễ dàng mở rộng, bảo trì và kiểm thử.

Dưới đây là cây thư mục chính của dự án (trong thư mục \texttt{app/src/main/java/com/example/stushare/}):

\begin{lstlisting}[basicstyle=\ttfamily\small]
com.example.stushare
|-- core/                # Các thành phần dùng chung (Core Module)
|   |-- data/            # Xử lý dữ liệu (Network, Database, Models)
|   |-- di/              # Cấu hình Hilt (Dependency Injection)
|   |-- domain/          # Business Logic (UseCases)
|   |-- utils/           # Các hàm tiện ích (DownloadHelper, Constants)
|   |-- workers/         # Tác vụ nền (DownloadWorker)
|
|-- features/            # Các màn hình chức năng (Feature Modules)
|   |-- auth/            # Đăng nhập, Đăng ký, Quên mật khẩu
|   |-- feature_home/    # Trang chủ, hiển thị danh sách tài liệu
|   |-- feature_search/  # Tìm kiếm và lọc tài liệu
|   |-- feature_upload/  # Đăng tải tài liệu mới
|   |-- feature_profile/ # Thông tin cá nhân, cài đặt
|   |-- feature_leaderboard/ # Bảng xếp hạng thành viên
|
|-- ui/theme/            # Cấu hình giao diện (Color, Type, Theme)
|-- MainActivity.kt      # Activity chính chứa Navigation Graph
|-- MainApplication.kt   # Class Application khởi tạo Hilt

\end{lstlisting}

\textbf{Giải thích các thành phần chính:}
\begin{itemize}
    \item \textbf{core:} Chứa các thành phần cốt lõi không phụ thuộc vào giao diện cụ thể, như \texttt{AppDatabase} (Room), \texttt{ApiService} (Retrofit) và các \texttt{Repository}.
    \item \textbf{features:} Mỗi thư mục con đại diện cho một tính năng lớn, bên trong chứa đầy đủ các lớp UI (Screen), ViewModel và State riêng biệt.
    \item \textbf{di (Dependency Injection):} Chứa các Module (\texttt{NetworkModule}, \texttt{DatabaseModule}, \texttt{FirebaseModule}) cung cấp đối tượng cho toàn bộ ứng dụng.
\end{itemize}

\section{Kết quả đạt được (Demo ứng dụng)}

Phần này trình bày giao diện thực tế của ứng dụng và các kịch bản kiểm thử (Test Cases) cho các chức năng chính.

\subsection{Chức năng Đăng nhập và Xác thực}

\begin{figure}[H]
    \centering
    \begin{minipage}{0.3\textwidth}
        \centering
        % Thay bằng ảnh chụp màn hình đăng nhập của bạn
        \includegraphics[width=\linewidth]{img/manhinhchao.jpg} 
        \caption{Màn hình Chào}
    \end{minipage}\hfill
    \begin{minipage}{0.3\textwidth}
        \centering
        \includegraphics[width=\linewidth]{img/dangnhapsdt.jpg}
        \caption{Đăng nhập Số điện thoạil}
    \end{minipage}\hfill
    \begin{minipage}{0.3\textwidth}
        \centering
        \includegraphics[width=\linewidth]{img/dangky.jpg}
        \caption{Đăng ký}
    \end{minipage}
\end{figure}

\textbf{Kịch bản kiểm thử 1: Đăng nhập thành công}
\begin{itemize}
    \item \textbf{Đầu vào:} Người dùng nhập Email và Mật khẩu đúng định dạng đã đăng ký.
    \item \textbf{Thao tác:} Nhấn nút "Đăng nhập".
    \item \textbf{Kết quả mong đợi:} Hệ thống xác thực qua Firebase, chuyển hướng vào Màn hình chính (Home) và lưu trạng thái đăng nhập.
    \item \textbf{Kết quả thực tế:} Đạt (Pass).
\end{itemize}

\subsection{Chức năng Trang chủ và Xem tài liệu}

\begin{figure}[H]
    \centering
    \begin{minipage}{0.45\textwidth}
        \centering
        \includegraphics[width=0.8\linewidth]{img/giaodientrangchu.jpg} 
        \caption{Giao diện Trang chủ}
    \end{minipage}\hfill
    \begin{minipage}{0.45\textwidth}
        \centering
        \includegraphics[width=0.8\linewidth]{img/chitiettailieu.jpg}
        \caption{Chi tiết tài liệu}
    \end{minipage}
\end{figure}

\textbf{Kịch bản kiểm thử 2: Tải xuống tài liệu}
\begin{itemize}
    \item \textbf{Đầu vào:} Chọn một tài liệu bất kỳ từ trang chủ.
    \item \textbf{Thao tác:} Nhấn nút "Tải xuống" (Download) ở màn hình chi tiết.
    \item \textbf{Xử lý hệ thống:} \texttt{DownloadWorker} chạy nền để tải file từ URL.
    \item \textbf{Kết quả mong đợi:} Hiển thị thông báo (Notification) tiến trình tải và thông báo "Tải xuống thành công" khi hoàn tất.
    \item \textbf{Kết quả thực tế:} Đạt (Pass).
\end{itemize}

\subsection{Chức năng Tìm kiếm}

\begin{figure}[H]
    \centering
    \includegraphics[width=5cm]{img/manhinhtimkiem.jpg} 
    \caption{Màn hình Tìm kiếm}
\end{figure}

\textbf{Kịch bản kiểm thử 3: Tìm kiếm theo từ khóa}
\begin{itemize}
    \item \textbf{Đầu vào:} Nhập từ khóa "Lập trình di động" vào thanh tìm kiếm.
    \item \textbf{Thao tác:} Nhấn Enter hoặc icon Search.
    \item \textbf{Kết quả mong đợi:} Danh sách hiển thị các tài liệu có chứa từ khóa trong tiêu đề hoặc mô tả. Nếu không có, hiển thị thông báo "Không tìm thấy kết quả".
    \item \textbf{Kết quả thực tế:} Đạt (Pass).
\end{itemize}

\subsection{Chức năng Cá nhân và Cài đặt}

\begin{figure}[H]
    \centering
    \begin{minipage}{0.45\textwidth}
        \centering
        \includegraphics[width=0.8\linewidth]{img/hosocanhan.jpg} 
        \caption{Hồ sơ cá nhân}
    \end{minipage}\hfill
    \begin{minipage}{0.45\textwidth}
        \centering
        \includegraphics[width=0.8\linewidth]{img/manhinhcaidat.jpg}
        \caption{Màn hình Cài đặt}
    \end{minipage}
\end{figure}

\textbf{Kịch bản kiểm thử 4: Đổi mật khẩu}
\begin{itemize}
    \item \textbf{Đầu vào:} Nhập mật khẩu cũ và mật khẩu mới (2 lần).
    \item \textbf{Thao tác:} Nhấn "Lưu thay đổi".
    \item \textbf{Kết quả mong đợi:} Hệ thống cập nhật mật khẩu trên Firebase Auth và thông báo thành công.
    \item \textbf{Kết quả thực tế:} Đạt (Pass).
\end{itemize}

\section{Đánh giá kết quả}

Sau quá trình xây dựng và kiểm thử, nhóm thực hiện tiến hành so sánh kết quả thực tế với các mục tiêu đề ra ban đầu:

\begin{table}[h!]
    \centering
    \caption{Bảng đánh giá mức độ hoàn thành}
    \begin{tabular}{|p{1cm}|p{6cm}|p{3cm}|p{4cm}|}
        \hline
        \textbf{STT} & \textbf{Mục tiêu ban đầu} & \textbf{Trạng thái} & \textbf{Ghi chú} \\
        \hline
        1 & Xây dựng giao diện người dùng hiện đại bằng Jetpack Compose. & Hoàn thành & Giao diện mượt mà, hỗ trợ Dark Mode. \\
        \hline
        2 & Tích hợp đăng nhập/đăng ký qua Firebase. & Hoàn thành & Hỗ trợ cả Google Sign-In. \\
        \hline
        3 & Chức năng tải lên và tải xuống tài liệu. & Hoàn thành & Sử dụng WorkManager xử lý tác vụ nền ổn định. \\
        \hline
        4 & Tìm kiếm và lọc tài liệu theo thẻ (Tag). & Hoàn thành & Tốc độ phản hồi nhanh. \\
        \hline
        5 & Hệ thống gợi ý tài liệu thông minh (AI). & Chưa hoàn thành & Đang ở mức hiển thị tài liệu mới nhất, chưa có thuật toán AI. \\
        \hline
    \end{tabular}
\end{table}

\textbf{Nhận xét chung:} Ứng dụng StuShare đã đáp ứng được khoảng 90\% các yêu cầu chức năng cốt lõi của một nền tảng chia sẻ tài liệu. Các chức năng hoạt động ổn định trên các thiết bị kiểm thử. Một số tính năng nâng cao (như AI Suggestion) sẽ được phát triển trong giai đoạn tiếp theo.
% Tệp: sections/09_ketluan.tex

% Dùng \chapter* để không đánh số chương
\chapter*{KẾT LUẬN VÀ KIẾN NGHỊ}

% Thêm dòng này để tiêu đề vẫn hiện trong Mục lục
\addcontentsline{toc}{chapter}{KẾT LUẬN VÀ KIẾN NGHỊ}

\label{ch:ketluan}

Sau quá trình nghiên cứu và thực hiện đồ án "Xây dựng ứng dụng chia sẻ tài liệu StuShare", nhóm thực hiện đã hoàn thành các mục tiêu đề ra ban đầu, xây dựng được một ứng dụng di động hoàn chỉnh phục vụ nhu cầu học tập của sinh viên. Dưới đây là tóm tắt các kết quả đạt được, những hạn chế còn tồn tại và phương hướng phát triển trong tương lai.

\section*{Kết luận} % Dùng \section* nếu muốn bỏ số 1, 2, 3 ở mục con luôn
Đồ án đã giải quyết được bài toán kết nối và chia sẻ tri thức trong cộng đồng sinh viên thông qua nền tảng di động. Cụ thể, nhóm đã đạt được những kết quả sau về mặt kỹ thuật và chức năng:

\begin{itemize}
    \item \textbf{Về mặt sản phẩm:}
    \begin{itemize}
        \item Xây dựng thành công ứng dụng Android hoạt động ổn định với giao diện hiện đại, trải nghiệm người dùng (UX) mượt mà nhờ sử dụng \textbf{Jetpack Compose}.
        \item Hệ thống xác thực người dùng hoàn chỉnh (Đăng ký, Đăng nhập, Quên mật khẩu) tích hợp \textbf{Firebase Authentication} và Google Sign-In.
        \item Tính năng quản lý tài liệu: Cho phép người dùng tải lên (Upload) và tải xuống (Download) tài liệu (PDF, Word) thông qua \textbf{WorkManager} để xử lý tác vụ nền hiệu quả.
        \item Tính năng tìm kiếm mạnh mẽ: Hỗ trợ tìm kiếm tài liệu theo từ khóa và lọc theo thẻ (Tags) giúp sinh viên dễ dàng tiếp cận nội dung mong muốn.
        \item Xây dựng cộng đồng: Tính năng "Bảng xếp hạng" (Leaderboard) và "Yêu cầu tài liệu" giúp tăng tính tương tác và khuyến khích đóng góp.
    \end{itemize}

    \item \textbf{Về mặt công nghệ:}
    \begin{itemize}
        \item Áp dụng thành công kiến trúc \textbf{Clean Architecture} kết hợp với mô hình \textbf{MVVM} (Model-View-ViewModel), giúp mã nguồn dễ bảo trì và mở rộng.
        \item Sử dụng \textbf{Hilt} để quản lý Dependency Injection, tối ưu hóa việc khởi tạo đối tượng.
        \item Kết hợp linh hoạt giữa cơ sở dữ liệu cục bộ (\textbf{Room Database}) để lưu lịch sử/cache và cơ sở dữ liệu đám mây (API/Firebase) để đồng bộ dữ liệu.
        \item Xử lý bất đồng bộ hiệu quả với \textbf{Kotlin Coroutines} và \textbf{Flow}.
    \end{itemize}
\end{itemize}

\section*{Hạn chế}
Bên cạnh những kết quả đạt được, ứng dụng StuShare vẫn còn một số hạn chế nhất định do giới hạn về thời gian và nguồn lực:

\begin{itemize}
    \item \textbf{Nền tảng hỗ trợ:} Hiện tại ứng dụng chỉ mới phát triển trên hệ điều hành Android, chưa có phiên bản cho iOS hoặc Web, làm hạn chế số lượng người dùng tiếp cận.
    \item \textbf{Kiểm duyệt nội dung:} Quy trình kiểm duyệt tài liệu vẫn còn phụ thuộc nhiều vào thao tác thủ công của quản trị viên, chưa tích hợp AI để tự động phát hiện nội dung vi phạm bản quyền hoặc không phù hợp.
    \item \textbf{Tính năng xem trước:} Người dùng cần phải tải tài liệu về máy mới có thể xem nội dung chi tiết, chưa hỗ trợ xem trước (Preview) trực tuyến để tiết kiệm băng thông.
    \item \textbf{Tương tác xã hội:} Các tính năng bình luận, thả cảm xúc (Reaction) chi tiết vào từng tài liệu chưa được hoàn thiện sâu.
\end{itemize}

\section*{Hướng phát triển}
Để hoàn thiện sản phẩm và nâng cao trải nghiệm người dùng, nhóm đề xuất các hướng phát triển trong tương lai như sau:

\begin{itemize}
    \item \textbf{Đa nền tảng:} Nghiên cứu sử dụng Kotlin Multiplatform (KMP) để mở rộng ứng dụng sang nền tảng iOS mà không cần viết lại toàn bộ mã nguồn.
    \item \textbf{Tích hợp AI/Machine Learning:} 
    \begin{itemize}
        \item Xây dựng hệ thống gợi ý tài liệu thông minh dựa trên lịch sử tìm kiếm và chuyên ngành của sinh viên.
        \item Tích hợp OCR để trích xuất văn bản từ ảnh chụp tài liệu, hỗ trợ tìm kiếm nội dung bên trong file ảnh.
    \end{itemize}
    \item \textbf{Tối ưu hóa trải nghiệm:} Tích hợp trình đọc PDF/Docx ngay trong ứng dụng (In-app Viewer) để người dùng xem nhanh tài liệu.
    \item \textbf{Mở rộng hệ sinh thái:} Phát triển thêm tính năng "Góc học tập" cho phép tạo nhóm thảo luận trực tuyến và chia sẻ lộ trình học tập.
\end{itemize}

\vspace{1cm}
\noindent Nhóm sinh viên xin chân thành cảm ơn sự hướng dẫn tận tình của Giảng viên hướng dẫn và các thầy cô trong khoa đã tạo điều kiện để đồ án được hoàn thành tốt đẹp.

\printbibliography[heading=bibintoc, title={TÀI LIỆU THAM KHẢO}]
\appendix
% Tệp: sections/09_phuluc.tex
% \appendix đã được gọi trong main.tex
% Chương đầu tiên ở đây sẽ là "Phụ lục A"

\chapter{BẢNG KHẢO SÁT}
Nội dung phụ lục A...

\chapter{SOURCE CODE QUAN TRỌNG}
Nội dung phụ lục B...

\end{document}