% Tệp: sections/07_chuong2.tex

\chapter{PHÂN TÍCH VÀ THIẾT KẾ HỆ THỐNG}
\label{ch:phantichthietke}

Chương này trình bày chi tiết về các yêu cầu chức năng và phi chức năng của hệ thống, đồng thời mô tả các bản thiết kế về Use Case, luồng hoạt động, tương tác và cơ sở dữ liệu của ứng dụng StuShare.

\section{Phân tích yêu cầu hệ thống}

\subsection{Yêu cầu chức năng} 
Dựa trên khảo sát nhu cầu thực tế của sinh viên, hệ thống StuShare cần đáp ứng các nhóm chức năng chính sau:

\begin{itemize}
    \item \textbf{Nhóm chức năng Xác thực \& Tài khoản:}
    \begin{itemize}
        \item Đăng ký tài khoản mới (Email, Số điện thoại).
        \item Đăng nhập hệ thống (Email/Pass, Google Sign-In).
        \item Quên mật khẩu và khôi phục qua Email.
        \item Cập nhật thông tin cá nhân (Avatar, Tên, Chuyên ngành).
        \item Đổi mật khẩu và bảo mật tài khoản.
    \end{itemize}

    \item \textbf{Nhóm chức năng Quản lý Tài liệu:}
    \begin{itemize}
        \item Xem danh sách tài liệu mới nhất, tài liệu đề thi.
        \item Xem chi tiết thông tin tài liệu (Tên, mô tả, tác giả, lượt tải).
        \item Tải lên tài liệu mới (Upload file PDF, Word) kèm tiêu đề và thẻ (Tag).
        \item Tải xuống tài liệu về thiết bị.
    \end{itemize}

    \item \textbf{Nhóm chức năng Tương tác \& Tìm kiếm:}
    \begin{itemize}
        \item Tìm kiếm tài liệu theo từ khóa.
        \item Lọc tài liệu theo thẻ (Tag).
        \item Xem bảng xếp hạng thành viên tích cực.
        \item Gửi yêu cầu tìm tài liệu (Request Document).
    \end{itemize}
\end{itemize}

\subsection{Yêu cầu phi chức năng}
\begin{itemize}
    \item \textbf{Hiệu năng:} Ứng dụng phản hồi nhanh, thời gian tải danh sách tài liệu dưới 2 giây trong điều kiện mạng ổn định.
    \item \textbf{Bảo mật:} Mật khẩu người dùng được mã hóa. Sử dụng Token để xác thực phiên làm việc. Dữ liệu cá nhân được bảo vệ theo quy định.
    \item \textbf{Giao diện (UI/UX):} Thiết kế hiện đại, thân thiện, dễ sử dụng, hỗ trợ giao diện Sáng/Tối (Light/Dark Mode).
    \item \textbf{Độ tin cậy:} Hệ thống hoạt động ổn định 24/7, có cơ chế xử lý lỗi khi mất kết nối mạng (Offline Mode cơ bản).
\end{itemize}

\section{Thiết kế Use Case}

\subsection{Sơ đồ Use Case tổng quát}
Sơ đồ dưới đây mô tả tổng quan các tác nhân và chức năng của hệ thống StuShare.

\begin{figure}[H]
    \centering
    % height=0.4\textheight: Giới hạn chiều cao để 1 trang chứa được 2 hình
    \includegraphics[width=1.1\textwidth, height=0.6\textheight, keepaspectratio]{img/usecase/usecase2.png} 
    \caption{Sơ đồ Use Case tổng quát hệ thống StuShare}
    \label{fig:usecase_tongquat}
\end{figure}

\subsection{Đặc tả các Use Case chính}

\begin{table}[H]
    \centering
    \caption{Đặc tả Use Case "Đăng nhập"}
    \begin{tabular}{|p{4cm}|p{10cm}|}
        \hline
        \textbf{Tên Use Case} & \textbf{Đăng nhập hệ thống} \\
        \hline
        \textbf{Tác nhân} & Sinh viên (User) \\
        \hline
        \textbf{Mục đích} & Cho phép người dùng truy cập vào các chức năng yêu cầu xác thực. \\
        \hline
        \textbf{Luồng sự kiện chính} & 
        1. User mở ứng dụng, chọn chức năng Đăng nhập. \newline
        2. Hệ thống hiển thị form nhập Email/Mật khẩu. \newline
        3. User nhập thông tin và nhấn "Đăng nhập". \newline
        4. Hệ thống kiểm tra thông tin qua Firebase Auth. \newline
        5. Nếu đúng, chuyển hướng vào Trang chủ. \\
        \hline
        \textbf{Ngoại lệ} & Nhập sai mật khẩu quá 5 lần -> Khóa tạm thời. \\
        \hline
    \end{tabular}
\end{table}

\begin{table}[H]
    \centering
    \caption{Đặc tả Use Case "Tải lên tài liệu"}
    \begin{tabular}{|p{4cm}|p{10cm}|}
        \hline
        \textbf{Tên Use Case} & \textbf{Tải lên tài liệu} \\
        \hline
        \textbf{Tác nhân} & Sinh viên (User) \\
        \hline
        \textbf{Điều kiện tiên quyết} & User đã đăng nhập thành công. \\
        \hline
        \textbf{Luồng sự kiện chính} & 
        1. User chọn nút "Upload" từ màn hình chính. \newline
        2. User chọn file từ thiết bị và điền thông tin (Tiêu đề, Mô tả). \newline
        3. User nhấn "Đăng tải". \newline
        4. Hệ thống upload file lên Storage và lưu metadata vào Database. \newline
        5. Thông báo thành công và cộng điểm cho User. \\
        \hline
    \end{tabular}
\end{table}

\section{Thiết kế luồng hoạt động (Activity Diagram)}
Dưới đây là các sơ đồ hoạt động mô tả quy trình nghiệp vụ của 9 chức năng chính trong hệ thống.

% 2.3.1
\subsection{Sơ đồ hoạt động chức năng Đăng ký}
\begin{figure}[H] 
    \centering 
    \includegraphics[width=0.85\textwidth, height=0.4\textheight, keepaspectratio]{img/activity/dangki.png} 
    \caption{Activity Diagram - Đăng ký tài khoản} 
\end{figure}

% 2.3.2
\subsection{Sơ đồ hoạt động chức năng Đăng nhập}
\begin{figure}[H] 
    \centering 
    \includegraphics[width=0.85\textwidth, height=0.4\textheight, keepaspectratio]{img/activity/dangnhap.png} 
    \caption{Activity Diagram - Đăng nhập} 
\end{figure}

% 2.3.3
\subsection{Sơ đồ hoạt động chức năng Tìm kiếm tài liệu}
\begin{figure}[H] 
    \centering 
    \includegraphics[width=0.85\textwidth, height=0.4\textheight, keepaspectratio]{img/activity/timkiemtailieu.png} 
    \caption{Activity Diagram - Tìm kiếm tài liệu} 
\end{figure}

% 2.3.4
\subsection{Sơ đồ hoạt động chức năng Báo cáo vi phạm}
\begin{figure}[H] 
    \centering 
    \includegraphics[width=0.85\textwidth, height=0.4\textheight, keepaspectratio]{img/activity/baocaovipham.png} 
    \caption{Activity Diagram - Báo cáo vi phạm} 
\end{figure}

% 2.3.5
\subsection{Sơ đồ hoạt động chức năng Tải xuống tài liệu}
\begin{figure}[H] 
    \centering 
    \includegraphics[width=0.85\textwidth, height=0.4\textheight, keepaspectratio]{img/activity/taivetailieu.png} 
    \caption{Activity Diagram - Tải xuống tài liệu} 
\end{figure}

% 2.3.6
\subsection{Sơ đồ hoạt động chức năng Tải lên tài liệu}
\begin{figure}[H] 
    \centering 
    \includegraphics[width=0.85\textwidth, height=0.4\textheight, keepaspectratio]{img/activity/tailentailieu.png} 
    \caption{Activity Diagram - Tải lên tài liệu} 
\end{figure}

% 2.3.7
\subsection{Sơ đồ hoạt động chức năng Cập nhật bảng xếp hạng}
\begin{figure}[H] 
    \centering 
    \includegraphics[width=0.85\textwidth, height=0.4\textheight, keepaspectratio]{img/activity/capnhatbangxemhang.png} 
    \caption{Activity Diagram - Cập nhật bảng xếp hạng} 
\end{figure}

% 2.3.8
\subsection{Sơ đồ hoạt động chức năng Tạo yêu cầu tài liệu}
\begin{figure}[H] 
    \centering 
    \includegraphics[width=0.85\textwidth, height=0.4\textheight, keepaspectratio]{img/activity/yeucautailieu.png} 
    \caption{Activity Diagram - Tạo yêu cầu tài liệu} 
\end{figure}

% 2.3.9
\subsection{Sơ đồ hoạt động chức năng Quên mật khẩu}
\begin{figure}[H] 
    \centering 
    \includegraphics[width=0.85\textwidth, height=0.4\textheight, keepaspectratio]{img/activity/quenmatkhau.png} 
    \caption{Activity Diagram - Quên mật khẩu} 
\end{figure}


\section{Thiết kế tương tác (Sequence Diagram)}
Mục này trình bày chi tiết sự tương tác giữa các đối tượng trong hệ thống thông qua các sơ đồ tuần tự.

% 1. Báo cáo vi phạm
\subsection{Sơ đồ tuần tự Báo cáo vi phạm}
\begin{figure}[H] 
    \centering 
    \includegraphics[width=0.85\textwidth, height=0.4\textheight, keepaspectratio]{img/Sequence/baocaovipham.png} 
    \caption{Sequence Diagram - Báo cáo vi phạm} 
\end{figure}

% 2. Bỏ lưu tài liệu
\subsection{Sơ đồ tuần tự Bỏ lưu tài liệu}
\begin{figure}[H] 
    \centering 
    \includegraphics[width=0.85\textwidth, height=0.4\textheight, keepaspectratio]{img/Sequence/boluutailieu.png} 
    \caption{Sequence Diagram - Bỏ lưu tài liệu} 
\end{figure}

% 3. Chỉnh sửa thông tin
\subsection{Sơ đồ tuần tự Chỉnh sửa thông tin}
\begin{figure}[H] 
    \centering 
    \includegraphics[width=0.85\textwidth, height=0.4\textheight, keepaspectratio]{img/Sequence/chinhsuathongtin.png} 
    \caption{Sequence Diagram - Chỉnh sửa thông tin} 
\end{figure}

% 4. Đăng ký thành công
\subsection{Sơ đồ tuần tự Đăng ký thành công}
\begin{figure}[H] 
    \centering 
    \includegraphics[width=0.85\textwidth, height=0.4\textheight, keepaspectratio]{img/Sequence/dangkithanhcong.png} 
    \caption{Sequence Diagram - Đăng ký thành công} 
\end{figure}

% 5. Đăng nhập thành công
\subsection{Sơ đồ tuần tự Đăng nhập thành công}
\begin{figure}[H] 
    \centering 
    \includegraphics[width=0.85\textwidth, height=0.4\textheight, keepaspectratio]{img/Sequence/dangnhapthanhcong.png} 
    \caption{Sequence Diagram - Đăng nhập thành công} 
\end{figure}

% 6. Đăng nhập thất bại
\subsection{Sơ đồ tuần tự Đăng nhập thất bại}
\begin{figure}[H] 
    \centering 
    \includegraphics[width=0.85\textwidth, height=0.4\textheight, keepaspectratio]{img/Sequence/dangnhapthatbai.png} 
    \caption{Sequence Diagram - Đăng nhập thất bại} 
\end{figure}

% 7. Đăng xuất
\subsection{Sơ đồ tuần tự Đăng xuất}
\begin{figure}[H] 
    \centering 
    \includegraphics[width=0.85\textwidth, height=0.4\textheight, keepaspectratio]{img/Sequence/dangxuat.png} 
    \caption{Sequence Diagram - Đăng xuất} 
\end{figure}

% 8. Đổi mật khẩu
\subsection{Sơ đồ tuần tự Đổi mật khẩu}
\begin{figure}[H] 
    \centering 
    \includegraphics[width=0.85\textwidth, height=0.4\textheight, keepaspectratio]{img/Sequence/doimk.png} 
    \caption{Sequence Diagram - Đổi mật khẩu} 
\end{figure}

% 9. Lưu tài liệu
\subsection{Sơ đồ tuần tự Lưu tài liệu}
\begin{figure}[H] 
    \centering 
    \includegraphics[width=0.85\textwidth, height=0.4\textheight, keepaspectratio]{img/Sequence/luutailieu.png} 
    \caption{Sequence Diagram - Lưu tài liệu} 
\end{figure}

% 10. Quên mật khẩu
\subsection{Sơ đồ tuần tự Quên mật khẩu}
\begin{figure}[H] 
    \centering 
    \includegraphics[width=0.85\textwidth, height=0.4\textheight, keepaspectratio]{img/Sequence/quyenmk.png} 
    \caption{Sequence Diagram - Quên mật khẩu} 
\end{figure}

% 11. Tải tài liệu thành công
\subsection{Sơ đồ tuần tự Tải tài liệu thành công}
\begin{figure}[H] 
    \centering 
    \includegraphics[width=0.85\textwidth, height=0.4\textheight, keepaspectratio]{img/Sequence/tailenthanhcong.png} 
    \caption{Sequence Diagram - Tải tài liệu thành công} 
\end{figure}

% 12. Tải tài liệu không thành công
\subsection{Sơ đồ tuần tự Tải tài liệu không thành công}
\begin{figure}[H] 
    \centering 
    \includegraphics[width=0.85\textwidth, height=0.4\textheight, keepaspectratio]{img/Sequence/tailen0thanhcong.png} 
    \caption{Sequence Diagram - Tải tài liệu không thành công} 
\end{figure}

% 13. Tải về tài liệu
\subsection{Sơ đồ tuần tự Tải về tài liệu}
\begin{figure}[H] 
    \centering 
    \includegraphics[width=0.85\textwidth, height=0.4\textheight, keepaspectratio]{img/Sequence/taivetailieu.png} 
    \caption{Sequence Diagram - Tải về tài liệu} 
\end{figure}

% 14. Tìm kiếm - có kết quả
\subsection{Sơ đồ tuần tự Tìm kiếm - có kết quả}
\begin{figure}[H] 
    \centering 
    \includegraphics[width=0.85\textwidth, height=0.4\textheight, keepaspectratio]{img/Sequence/timkiemcokq.png} 
    \caption{Sequence Diagram - Tìm kiếm có kết quả} 
\end{figure}

% 15. Yêu cầu tài liệu
\subsection{Sơ đồ tuần tự Yêu cầu tài liệu}
\begin{figure}[H] 
    \centering 
    \includegraphics[width=0.85\textwidth, height=0.4\textheight, keepaspectratio]{img/Sequence/yeucautailieu.png} 
    \caption{Sequence Diagram - Yêu cầu tài liệu} 
\end{figure}


\section{Thiết kế Cơ sở dữ liệu}

\subsection{Sơ đồ quan hệ thực thể (ERD)}
Mô hình dữ liệu của hệ thống bao gồm các thực thể chính: Người dùng, Tài liệu, Thông báo và Yêu cầu tài liệu.
\begin{figure}[H]
    \centering
    \includegraphics[width=0.9\textwidth, height=0.4\textheight, keepaspectratio]{img/db1.jpg} 
    \caption{Sơ đồ quan hệ thực thể (ERD)}
\end{figure}

\subsection{Mô tả chi tiết các bảng dữ liệu}
Dựa trên kiến trúc Room Database và Firebase Firestore đã cài đặt, dưới đây là chi tiết các bảng:

\subsubsection{Bảng Người dùng (UserEntity / users collection)}
Lưu trữ thông tin cá nhân và vai trò của người dùng trong hệ thống.
\begin{table}[H]
    \centering
    \begin{tabular}{|l|l|l|p{5cm}|}
        \hline
        \textbf{Tên trường} & \textbf{Kiểu dữ liệu} & \textbf{Ràng buộc} & \textbf{Mô tả} \\
        \hline
        id & String & PK & Mã định danh duy nhất (UID từ Firebase) \\
        \hline
        email & String & Unique, Not Null & Địa chỉ email đăng nhập \\
        \hline
        fullName & String & Not Null & Họ và tên hiển thị \\
        \hline
        studentId & String & Nullable & Mã số sinh viên \\
        \hline
        avatarUrl & String & Nullable & Đường dẫn ảnh đại diện \\
        \hline
        points & Int & Default 0 & Điểm thưởng đóng góp \\
        \hline
    \end{tabular}
    \mysource{Mã nguồn lớp UserEntity.kt}
\end{table}

\subsubsection{Bảng Tài liệu (Document / documents collection)}
Lưu trữ metadata của các tài liệu được chia sẻ.
\begin{table}[H]
    \centering
    \begin{tabular}{|l|l|l|p{5cm}|}
        \hline
        \textbf{Tên trường} & \textbf{Kiểu dữ liệu} & \textbf{Ràng buộc} & \textbf{Mô tả} \\
        \hline
        id & String & PK & Mã tài liệu \\
        \hline
        title & String & Not Null & Tiêu đề tài liệu \\
        \hline
        description & String & Nullable & Mô tả ngắn gọn \\
        \hline
        fileUrl & String & Not Null & Link tải file (Firebase Storage) \\
        \hline
        fileType & String & Not Null & Định dạng (PDF, DOCX...) \\
        \hline
        authorId & String & FK (User) & Người đăng tải \\
        \hline
        downloadCount & Int & Default 0 & Số lượt tải xuống \\
        \hline
    \end{tabular}
    \mysource{Mã nguồn lớp Document.kt}
\end{table}

\subsubsection{Bảng Thông báo (NotificationEntity)}
Lưu trữ các thông báo gửi đến người dùng.
\begin{table}[H]
    \centering
    \begin{tabular}{|l|l|l|p{5cm}|}
        \hline
        \textbf{Tên trường} & \textbf{Kiểu dữ liệu} & \textbf{Ràng buộc} & \textbf{Mô tả} \\
        \hline
        id & Int & PK, Auto Inc & Mã thông báo (Local DB) \\
        \hline
        title & String & Not Null & Tiêu đề thông báo \\
        \hline
        message & String & Not Null & Nội dung thông báo \\
        \hline
        isRead & Boolean & Default False & Trạng thái đã xem \\
        \hline
        timestamp & Long & Not Null & Thời gian nhận \\
        \hline
    \end{tabular}
    \mysource{Mã nguồn lớp NotificationEntity.kt}
\end{table}

\section{Thiết kế Giao diện}
Hệ thống sử dụng công nghệ \textbf{Jetpack Compose} để xây dựng giao diện người dùng (UI), đảm bảo tính hiện đại và mượt mà.
Các màn hình chính bao gồm:
\begin{itemize}
    \item \textbf{Màn hình Auth:} Login, Register với các trường nhập liệu và validate lỗi.
    \item \textbf{Màn hình Home:} Sử dụng LazyColumn để hiển thị danh sách tài liệu cuộn vô tận.
    \item \textbf{Màn hình Detail:} Hiển thị thông tin chi tiết và nút Download Floating Action Button.
\end{itemize}

\begin{figure}[H]
    \centering
    \includegraphics[width=0.9\textwidth, height=0.4\textheight, keepaspectratio]{img/mau.png} 
    \caption{Sơ đồ luồng màn hình (Screen Flow)}
\end{figure}